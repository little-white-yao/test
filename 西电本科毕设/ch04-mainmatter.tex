%% ----------------------------------------------------------------------
%% START OF FILE
%% ----------------------------------------------------------------------
%% 
%% Filename: ch04-mainmatter.tex
%% Author: Fred Qi
%% Created: 2012-12-26 09:20:28(+0800)
%% 
%% ----------------------------------------------------------------------
%%% CHANGE LOG
%% ----------------------------------------------------------------------
%% Last-Updated: 2015-04-11 14:11:37(+0300) [by Fred Qi]
%%     Update #: 19
%% ----------------------------------------------------------------------
\ifx\allfiles\undefined
\documentclass[bachelor,print,msfonts]{xduthesis}

% 这里是可能用到的其他宏包,根据自己论文撰写的需要添加
\usepackage{listings}
\usepackage{overpic}
\lstset{language=TeX, basicstyle=\ttfamily}

\newcommand{\figref}[1]{图\ref{#1}}
\newcommand{\sArt}{state-of-the-art}
\newcommand{\wyh}[1]{{\textcolor{blue}{#1}}}
\newcommand{\secref}[1]{Sec. \ref{#1}}
\newcommand{\tabref}[1]{Tab.~\ref{#1}}
\newcommand{\thudot}[1]{} % thubib.bst 定义每条参考文献最后的点\thudot
\graphicspath{{./}{./img/}{./fig/}{./image/}{./figure/}{./picture/}}
\newcommand{\bs}{\boldsymbol}
\usepackage{float}
\setcounter{chapter}{3}
\begin{document}
\mainmatter
\fi

\chapter{总结与展望}
\label{cha:mainmatter}
\section{总结}
本文基于MPGE提出的TMC13v20测试平台,对Trisoup几何压缩算法进行了改进,主要体现在对熵编码过程所用到的上下文的顺序进行了调整,整体性能效果没达到预期目标,但是个别序列性能增益挺大,通过这一系列测试以及对结果的分析,发现了很多可能改进的方向。

文章首先介绍了GPCC中点云编解码的整体流程,其中对八叉树编码,Trisoup几何编码,RATH变换编码,Lifting预测变换编码等几何编码和属性编码的常用方法进行了详细介绍,还对点云压缩算法的性能评价标准进行了介绍;然后通过对Trisoup几何信息熵编码的上下文构造的介绍,提出其中可以进行改进的方向,紧接着分小节对上下文信息熵与条件熵的具体测量过程进行了介绍,最终得到了一套全新的局部最优的上下文模型;本文第四章介绍了本次实验在solid、dense、sparse、scant四类测试序列,几何有损、属性有损的条件下得到的实验结果,结果表明本文提出的根据信息熵大小决定首位上下文,再根据条件熵大小决定后续上下文的方法并不理想,D1、D2均有0.9\%的编码增益。
\section{展望}
通过对实验结果的分析,以及对动态OBUF的深入理解,得到以下改进方法:

1、目前所用到的各类上下文模型都不是二元取值的,都是由大于1bit位的信息构成,这样的上下文用于动态OBUF进行熵编码时,可能并不能够充分消除点云之间的空间相关性,甚至可能将熵编码的概率模型收敛至非理想值。这就导致最终的性能结果出现较大负增益。后续可以通过拆分某一上下文,然后交换该上下文多个bit位的位置来观察测试序列的压缩性能变化,从而判断是否可以从这一方向进行改进。

2、考虑到改进后的上下文模型对于某些序列存在着较大的增益,如表\ref{有增益序列}所示。如果能找出这些序列之间的一个共性,并且能够有效的代表某一类点云序列,那么我们通过设置自适应判断的方法,为不同类型的点云序列采用不同的上下文模型,这样我们得到的性能增益一定是可观的。

Trisoup几何压缩算法在Ofinno,小米,OPPO等公司的不断改进下,对于表面稠密点云展现出越来越好的压缩性能。截止目前,每次MPEG召开的国际会议上,都会有大量的有关Trisoup的改进提案涌现,同时针对表面连续的稠密点云(solid),MPEG新发布了一套几何压缩测试模型(Ges-TM)。这暗示着这类几何压缩算法在未来能带来更加理想的性能结果。


\ifx\allfiles\undefined
\end{document}
\fi
%% ----------------------------------------------------------------------
%%% END OF FILE 
%% ----------------------------------------------------------------------